\section*{Sommario}
In questa relazione, gli autori presentano un'applicazione che determina se regioni di un'immagine contengono malformazioni della tessitura al loro interno.
Verrà fatta una panoramica sulle immagini di tessiture. Successivamente saranno presentati i concetti teorici alla base dell'operatore \acf{LBP} ed infine verranno mostrati gli strumenti utilizzati per lo sviluppo ed i risultati ottenuti.


\newpage
\null 
\thispagestyle{empty}
\newpage


\section{Texture}

Una texture è una nozione generale che può avere molteplici significati. In particolare nel contesto della percezione umana una texture è una specifica struttura di superfici, formate da uno o più particolari elementi, che si ripetono nello spazio.
Questa ripetizione può riguardare:

\begin{itemize}
	\item la variazione locale di scala;
	\item l'orientazione;
	\item la distribuzione spaziale dei livelli di grigio o del colore\cite{Stockman:2001:CV:558008};
	\item o altre caratteristiche geometriche dell'elemento.
\end{itemize}

\noindent In figura \ref{fig:Texture} sono rappresentati alcuni esempi di immagini di tessiture. \\

\begin{figure}[ht]
\begin{center}
\includegraphics[width=.95\textwidth]{img/Texture}
\caption{ Immagini di tessiture.}
\label{fig:Texture}
\end{center}
\end{figure}


Una immagine è percepita contenente tessiture quando il numero di elementi che la compongono è sufficientemente numeroso.
Se il numero di elementi presenti nell'immagine è ridotto, allora questi sono percepiti come un gruppo di oggetti numerabili invece che una tessitura\cite{SlidePala}.

Alcune delle caratteristiche qualitative che possono essere utilizzate per differenziare le texture sono:

\begin{itemize}
\item fineness - coarseness,
\item smoothness,
\item granularity,
\item lineation,
\item directionality,
\item regularity - randomness
\end{itemize}

\noindent L'utilizzo di modelli di tessiture può trovare interessanti applicazioni nel contesto della:

\begin{itemize}
\item Segmentazione di immagini;
\item Classificazione di immagini.
\end{itemize}

La \textit{segmentazione} di immagini consiste nel suddividerle in regioni che presentano tessiture uniformi. La \textit{classificazione}, associa ad ogni regione ottenuta dal processo di segmentazione la sua classe di appartenenza.