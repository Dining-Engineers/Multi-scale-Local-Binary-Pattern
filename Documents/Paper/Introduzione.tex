\section{Introduzione}

\subsection{Description of region features}

Texture is a very general notion that can be
attributed to almost everything in nature.
For a human, the texture relates mostly to:
A specific, spatially repetitive (micro)structure of surfaces
formed by repeating a particular element or several
elements in different relative spatial positions.
Generally, the repetition involves local variations of scale,
orientation, or other geometric and optical features of the
elements.Texture
Texture is a property of regions: the texture of a
point is undefined
Texture involves the spatial distribution of gray
levels or colors
Texture in an image can be perceived at different
scales or levels of resolution
A region is perceived to have texture when the
number of primitive elements in the region is large. If
only a few primitive elements are present, then a
group of countable objects is perceived instead of a
textured imageTexture
It is almost impossible to provide a complete description of
textures in words
However, words capture some informal qualitative
features that can help
discriminate different
textures
fineness - coarseness,
smoothness,
granularity,
lineation,
directionality,
regularity - randomness


Texture models can be of interest for three
main purposes:
Segmentation: partition the input image
into regions of uniform texture


Classification: produce a classification map of the
input image where each uniform textured region
is identified with the texture class it belongs to


Synthesis: generate an image with a texture that
looks like a requested texture class